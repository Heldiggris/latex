\documentclass[a5paper, 11pt, twoside]{article}
\usepackage{ upgreek }
\usepackage[left=1.5cm,right=1cm,
top=1cm,bottom=2cm,bindingoffset=0cm]{geometry}
\usepackage[T1, T2A]{fontenc}                       % Поддержка русских букв
\usepackage[utf8]{inputenc}                     % Кодировка utf8
\usepackage[english, russian]{babel}
\usepackage{ amsmath }
\usepackage{ amssymb }
\usepackage{setspace}
\usepackage{mathtools} 


\usepackage[integrals]{wasysym}



\renewcommand{\baselinestretch}{0.8}


\pagestyle{empty}

\begin{document}


{\noindent\scriptsize{\bfseries 144}
\hfill{\textsection{ } 47. Криволинейные интегралы}\\ \\}
{\renewcommand{\baselinestretch}{5.00}
 \noindentОбозначим через $\upgamma_\upeta$ совокупность всех точек плоскости, находящихся от $\upgamma$ на расстоянии, не большем чем $\upeta$. Множество $\upgamma_\upeta$ ограничено, замкнуто (см. там же лемму 6) и $\upgamma_\upeta{\subset}$G.
\par В силу равномерной непрерывности функций x(t) и y(t) на отрез-\linebreak ке
[a, b] существует такое число $\updelta_0 > 0$, что для любых двух точек\linebreak $t' \in[a,b]$ и $t''\in[a, b]$, удовлетворяющих условию $|t' - t''| < \updelta_0$,\linebreak выполняется неравенство
}

{\baselineskip=7pt
$$\small\uprho(M', M'')=\sqrt{[x(t'')-x(t')]^2+[y(t'')-y(t')]^2}<\upeta, $$}
{\noindentгде $M' = (x(t'), y(t')), M'' = (x(t''), y(t''))$. Все точки отрезка\linebreak с концовками в точках M' и M'', очевидно, также находятся от точки M'\linebreak на расстоянии, не большем чем $\upeta$, и потому лежат в $\upgamma_\upeta$ и, следова-\linebreakтельно, в G. Поэтому, если мелкость $\updelta_\uptau$ разбиения $\uptau$ отрезка [a, b]\linebreak такова, что $\updelta_\uptau < \updelta_0$, то все точки ломанной $\uplambda_\uptau$ лежат в G, и для таких\linebreak разбиений $\uptau$ имеет смысл рассматривать интеграл $\int\limits_{\updelta_\uptau}{Pdx}+ Qdy$.
\par
}
{\baselineskip=5ptРассмотрим интегралы $\int\limits_\uptau{P} dx$ и $ \int\limits_{\updelta_\uptau}P dx.$ }
Положим \\
$x_i=x(t_i), y_i=y(t_i), P_i=P(x_i, y_i), \Updelta{x_i=x_i-x_{i-1}}, i=1, 2,  ..., i_0, \linebreak \upsigma_\uptau=\sum\limits_{i=1}^{i_0}P_i\Updelta{x_i}.$
{\baselineskip=0pt\par Как известно (см. п. 47.2, свойство 4),\\}
{\baselineskip=0pt$$\small\lim\limits_{\updelta_\uptau\to0}
\upsigma_\uptau =\int\limits_\uptau{P} dx. \eqno{(47.35)}$$}
\par Пусть, далее, $M_i=(x_i, y_i)$-вершины ломанной $\uplambda_\uptau$, тогда \\
{\baselineskip=0pt$$\small\int\limits_{\uplambda_\uptau}=\sum\limits_{i=1}^{i_0}\int\limits_{M_{i-1}  M_i}P dx.\eqno{(47.36)}$$}
{\noindent
С другой стороны, заметим, что (употребляя обозначение п. 47.2)
}
{\baselineskip=0pt$$\small\int\limits_{M_{i-1} M_i}dx=\int\limits_{M_{i-1} Mi}cos \upalpha ds=|M_{i-1}M_i|cos \upalpha=\Updelta{x_i},$$ }\\
поэтому\\
{\baselineskip=0pt$$\small\upsigma_\uptau=\sum\limits_i P_i \Updelta{x_i}=\sum\limits_i\int\limits_{M_{i-1} M_i}P_i dx. \eqno{(47.37)}$$}
{\noindent
Обозначая через $L_\uptau$ длину ломанной $\upalpha_\uptau$, через S-длину кривой $\upgamma$,\linebreak а через $\upomega(\updelta; P)$-модуль непрерывности функции P(x, y) на ограни- \linebreak ченном замкнутом множестве $\upgamma_\upeta$, из (47.36) и (47.37) получим \\
}
{\baselineskip=0pt$$\small|\int\limits_{\uplambda_\uptau} P dx-\upsigma_\uptau| \leqslant \sum\limits_i|\int\limits_{M_{i-1} M_i} |P-P_i|dx | \leqslant$$}
\newpage

\end{document}
